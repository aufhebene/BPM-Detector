\documentclass[12pt]{article}

% All the packages used.
\usepackage[margin=1.0in]{geometry}
\usepackage{pbox}
\usepackage{array}

\begin{document}

% Title page.
\begin{center}
  \vspace*{1in}
  \textbf{\LARGE{BPM Detector (Android)}} \\
  \large{Requirements Document \\
  Duc Dao \\
  CPE 491, Spring 2017 \\
  California Polytechnic State University, San Luis Obispo} \\
\end{center}

% Abstract. Must be around 100 to 200 words.
\begin{center}
  \vspace*{2in}
  \textbf{Abstract} \\
\end{center}
The BPM Detector is an Android application that detects the tempo or beats-per-minute (BPM) for a band conductor's conducting. Generally, the conductor moves their arms in a predictable pattern in order to maintain the band's synchronization, acting similarly to a metronome. The purpose of this application is to provide a tool for concert and marching band conductors to detect their precise tempo while conducting. This will help them determine how well they are coordinating the band and whether they need to adjust their conducting.  

\newpage

\tableofcontents

\newpage

% ========================================================================================================================
\section{Introduction}
\subsection{Purpose}
This document is created to discuss the specifications of the BPM Detector, an Android application that detects the beats-per-minute of a conductor's tempo while conducting. The main purpose of the BPM Detector is to aid conductors in seeing how well they are conducting the band and determine whether they need to make adjustments or not. This will benefit marching bands because these types of bands are generally more sensitive to tempo inconsistencies compared to other types of bands.
\subsection{Document Conventions}
This document is broken up into six main sections with the following headings:
\begin{enumerate}
	\item Introduction
    \item Overall Description
    \item External Interface Requirements
    \item Functional Requirements
    \item Nonfunctional Requirements
 \end{enumerate}
 These main sections each include subsections and minor subsections, which may contain descriptions, bulleted lists, and tables.
 
\begin{center}
\Large \textbf{1. Main Section} \\
\large \textbf{1.1 Subsection} \\
\normalsize \textbf{1.1.1 Minor Subsection} \\
\normalsize Body text \\
\end{center}

\subsection{Intended Audience}
The intended audience for this document includes:
\begin{itemize}
	\item Developer
    \item Advisors
    \item Application Testers
 \end{itemize}

\subsubsection{Developer}
This document will serve as a guide for myself, the developer of BPM Detector. It will primarily be used to aid in my design and implementation of this project. 

\subsubsection{Advisors}
Advisors include:
\begin{itemize}
	\item My senior project advisor, Professor Franz Kurfess
    \item All other resources that aid in my project which may include other professors, students, and other developers
 \end{itemize}

\subsubsection{Application Testers}
Application Testers are users that have knowledge and capability of conducting a song. This includes, but is not limited to:
\begin{itemize}
	\item Band directors
    \item Drum majors, student conductors
    \item Professors in the Music department
 \end{itemize}



% ========================================================================================================================
\section{Overall Description}
\subsection{Background}
In the context of music ensembles, conducting is the act of directing the group in order to synchronize various components of the band and the performance. Members rely on the conductor to initiate the piece/song that they are performing, set the beats-per-minute or tempo of the song, cue points of interest in the piece, and other musical duties. 
\\
\\
With regards to the tempo, there are no applications to automatically detect the conductor's BPM while they are conducting. Existing solutions on the Google Play Store have users tap on the screen to manually calculate the BPM and metronome applications only sound off the BPM by setting a certain tempo, or listening to the environment.        

\subsection{Product Functionality}
BPM Detector's will perform one major function: to automatically detect the conductor's tempo while they are conducting.

\subsection{User Class and Their Characteristics}
BPM Detector will be used by one general user class: conductors. Conductors' experiences varies from person to person but all of them generally have knowledge of:
\begin{itemize}
	\item Conducting patterns
    \item Feeling of tempos
    \item Technicalities of the song they are conducting
 \end{itemize}

\subsection{Operating Environment}
BPM Detector will be an Android application running on a smartwatch with at least Android Wear 1.5. The smartwatch requires a Bluetooth connection to an Android phone running at least 6.0 Marshmallow. The smartwatch must have the follow specifications:
\begin{itemize}
	\item Central Processing Unit (CPU): 1.2 GHz quad-core Qualcomm® Snapdragon™ 400 CPU (APQ 8026) or greater
    \item Graphics Processing Unit (GPU): Adreno 305 with 450MHz GPU or greater
    \item Memory: 2 GB or greater
    \item Storage: At least 100 MB of free storage
    \item Connectivity: Bluetooth 4.0 or greater
 \end{itemize}
 
 

% ========================================================================================================================
\section{External Interface Requirements}
\subsection{User Interfaces}
Users of BPM Detector will interact with two user interfaces (UI); the UI of the Android Wear smartwatch, and the UI of the Android smartphone. On both devices, users will be able to:
\begin{itemize}
	\item Start the BPM detection
    \item Stop the BPM detection
    \item View past detections
 \end{itemize}
 BPM Detector will follow Material Design, a design language developed by Google. 

\subsection{Hardware Interfaces}
BPM Detector will be installed on the smartwatch and on the Android smartphone it is connected to. Bluetooth connectivity on the phone must persist during the life of a detection session in order for the smartwatch to send information to the smartphone. 

\subsection{Software Interfaces}
Both devices will have the capacity to initiate the BPM detection. During detection, the smartwatch must be able to interpret data from the sensors of the smartwatch and output a number representing the BPM. The smartwatch must interface with the smartphone in order for the smartphone to receive the BPM. This will allow the conductor to see the tempo they are conducting at.



% ========================================================================================================================
\section{Functional Requirements}
BPM Detector functional requirements include the following use cases:
\begin{enumerate}
	\item Starting BPM detection
    \item Stopping BPM detection
    \item Viewing past detections
 \end{enumerate}

\subsection{Starting BPM Detection}
\begin{center}
  \begin{tabular}{ |p{2.5in}|p{3in}| }
  \hline
  \textbf{Use Case ID} & 1 \\[.25in] 
  \hline
  \textbf{Name} & Starting BPM Detection\\[.25in] 
  \hline 
  \textbf{Description} & The act of starting the tempo detection. \\[.25in] 
  \hline 
  \textbf{Preconditions} & User must have the smartwatch on their wrist and the application on the phone launched. \\
  \hline
  \textbf{Postconditions} & None \\[.25in] 
  \hline
  \textbf{Priority} & High \\[.25in] 
  \hline 
  \textbf{Frequency of Use} & High \\[.25in] 
  \hline 
  \textbf{Normal Course} & User commences tempo detection. \\[.25in] 
  \hline 
  \textbf{Actor Action} \newline \newline 
  1. User selects "Start" on either the smartwatch or smartphone.
  & \textbf{System Responses} \newline \newline
  2. Smartwatch receives initiation and tells sensors to retrieve information in real-time. \newline
  3. Smartwatch interprets sensor information and calculates a BPM number. \newline
  4. Smartwatch sends BPM number to smartphone. \\[.25in] 
  \hline
  \end{tabular}
\end{center}

\subsection{Stopping BPM Detection}
\begin{center}
  \begin{tabular}{ |p{2.5in}|p{3in}| }
  \hline
  \textbf{Use Case ID} & 2 \\[.25in] 
  \hline
  \textbf{Name} & Stopping BPM Detection \\[.25in] 
  \hline 
  \textbf{Description} & The act of stopping the tempo detection. \\[.25in] 
  \hline 
  \textbf{Preconditions} & User started the application and initiated detection. Smartwatch is currently detecting BPM. \\
  \hline
  \textbf{Postconditions} & Smartwatch has stopped detecting BPM. \\[.25in] 
  \hline
  \textbf{Priority} & High \\[.25in] 
  \hline 
  \textbf{Frequency of Use} & High \\[.25in] 
  \hline 
  \textbf{Normal Course} & User ceases tempo detection. \\[.25in] 
  \hline 
  \textbf{Actor Action} \newline \newline 
  1. User selects "Stop" on either the smartwatch or smartphone.
  & \textbf{System Responses} \newline \newline
  2. Smartwatch receives cease selection and tells sensors to stop retrieving information in real-time. \newline
  3. Smartphone saves information of the duration of the detection as history for future viewing. \\[.25in] 
  \hline
  \end{tabular}
\end{center}
  
\subsection{Viewing Past Detections}
\begin{center}  
  \begin{tabular}{ |p{2.5in}|p{3in}| }
  \hline
  \textbf{Use Case ID} & 3 \\[.25in] 
  \hline
  \textbf{Name} & Viewing Past Detections \\[.25in] 
  \hline 
  \textbf{Description} & Viewing detections user made in the past. \\[.25in] 
  \hline 
  \textbf{Preconditions} & 
     User has done at least one detection and it saved correctly. Viewing past detections can only be done on the smartphone. \\
  \hline
  \textbf{Postconditions} & None \\[.25in] 
  \hline
  \textbf{Priority} & Low \\[.25in] 
  \hline 
  \textbf{Frequency of Use} & Medium \\[.25in] 
  \hline 
  \textbf{Normal Course} & User views past detections. \\[.25in] 
  \hline 
  \textbf{Actor Action} \newline \newline 
  1. User selects "View History" on smartphone. \newline \newline
  3. User selects which past detection they want to view. \newline
  & \textbf{System Responses} \newline \newline
  2. Smartphone receives selection and takes user to a listing on past detections. \newline \newline
  4. Smartphone displays past detection the user has selected and displays information associated with that detection. \\[.25in] 
  \hline
  \end{tabular}
\end{center}



% ========================================================================================================================
\section{Non-Functional Requirements}
\subsection{Performance Requirements}
All actions performed in the application must take a reasonable amount of time; the faster, the better. The goal is to keep all requests made (starting, stopping, viewing) under 5 seconds in terms of response time. To achieve this, the developer must implement the application with consideration to the smartwatch's limited resources.

\subsection{Safety Requirements}
During detection, the smartwatch's and smartphone's temperature cannot exceed 110 °F. To prevent this, the developer must be able to utilize the smartwatch's limited resources efficiently.



\end{document}