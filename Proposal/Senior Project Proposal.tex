\documentclass[12pt]{article}

% All the packages used.
\usepackage[margin=1.0in]{geometry}
\usepackage{pbox}
\usepackage{array}

\begin{document}

% Title page.
\begin{center}
  \vspace*{1in}
  \textbf{\LARGE{BPM Detector (Android)}} \\
  \large{Senior Project Proposal \\
  Duc Dao \\
  CPE 491, Spring 2017 \\
  California Polytechnic State University, San Luis Obispo} \\
\end{center}

% Abstract. Must be around 100 to 200 words.
\begin{center}
  \vspace*{2in}
  \textbf{Abstract} \\
\end{center}
The BPM Detector is an Android application that detects the tempo or beats-per-minute (BPM) for a band conductor's conducting. Generally, the conductor moves their arms in a predictable pattern in order to maintain the band's synchronization, acting similarly to a metronome. The purpose of this application is to provide a tool for concert and marching band conductors to detect their precise tempo while conducting. This will help them determine how well they are coordinating the band and whether they need to adjust their conducting.  

\newpage

\section{Problem}
In marching and concert bands, there needs to be some sort of method to synchronize all the members of the band so that they play, rest, and breathe at the appropriate time. This synchronization is typically done by a person such as a conductor, director, or drum major. The conductor synchronizes the band by moving their arm(s) in a predictable pattern in order to maintain the music's tempo or beats-per-minute (BPM).
\\
\\
Currently, there are no applications to automatically detect the conductor's BPM while they are conducting. Existing solutions on the Google Play Store have users tap on the screen to manually calculate the BPM and metronome applications only sound off the BPM by setting a certain tempo, or listening to the environment.  

\section{Solution}

I am proposing an Android Wear application that would automatically detect the BPM of the conductor's conducting pattern. The application would utilize the smartwatch's accelerator and gyroscope to calculate the BPM in real-time. I will provisionally call this solution the \textbf{BPM Detector}.

\section{General Scope of the Work}

\subsection{Milestones and Tentative Schedule}
\begin{center}
  \begin{tabular}{ |c|c|p{4in}| } 
  \hline 
  \multicolumn{3}{|c|}{\textbf{Spring Quarter 2017}} \\
  \hline
  \textbf{Week} & \textbf{Milestone} & \textbf{Deliverables} \\ 
  \hline
  1 & Background Research & \\
  \hline 
  2 & Background Research & \\
  \hline 
  3 & Requirements & \\
  \hline
  4 & Requirements & Requirements Document \\
  \hline
  5 & High-Level Design & \\
  \hline 
  6 & High-Level Design & High-Level Design Document\\
  \hline 
  7 & Iteration 1 & \\
  \hline 
  8 & Iteration 1 & \\
  \hline 
  9 & Iteration 1 & \\
  \hline 
  10 & Iteration 1 & Progress Report\\
  \hline
  \end{tabular}

  \begin{tabular}{ |c|c|p{4in}| } 
  \hline 
  \multicolumn{3}{|c|}{\textbf{Fall Quarter 2017}} \\
  \hline
  \textbf{Week} & \textbf{Milestone} & \textbf{Deliverables} \\
  \hline
  1 & Iteration 2 & \\
  \hline 
  2 & Iteration 2 & \\
  \hline 
  3 & Iteration 3 & \\
  \hline
  4 & Iteration 3 & \\ 
  \hline 
  5 & Iteration 4 & \\
  \hline 
  6 &  Iteration 4 & \\
  \hline 
  7 & Iteration 5 & \\
  \hline 
  8 & Iteration 5 & \\
  \hline 
  9 & Iteration 6 & \\
  \hline 
  10 & Project Demonstration & Final Report\\ 
  \hline
  \end{tabular}
\newline
\end{center}

\subsection{Deliverables}
\begin{itemize}
  \item \textbf{Requirements Document}: Requirements needed for the project.
  \item \textbf{High-Level Design Document}: Architecture design of the project.
  \item \textbf{Progress Report}: Progress made during the first half of the project.
  \item \textbf{Final Report}: Document summarizing the overall project.
\end{itemize}

\section{Satisfying Senior Project Objectives}

\subsection{Independence and Ownership}

As the sole owner, this gives me the freedom to dictate the direction of the project such as designing the system and its implementation. This freedom holds me accountable for the success or failure of this project, and the knowledge of using the Android framework is a key factor in this outcome.

\subsection{Background Research}

I have never worked with the Android framework extensively, so learning how to utilize it will be the most substantial challenge I have to face. In addition, I will need to fundamentally understand the various sensors of my second generation Moto 360 as that will be a key factor in this project.  

\subsection{Creativity}

This project was conceived to primarily provide support and accountability to conductor's of marching bands. The creativity of the BPM Detector lies in the usage of the Android Wear platform, a platform that still remains in its infancy. I hope that this solution gives conductors the tool to accurately measure the success of their conducting. I believe the usage of Android Wear is a unique solution to the problem of BPM detection in conducting.  

\end{document}